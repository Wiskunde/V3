%\section{Referenties}

\begin{thebibliography}{8}
\bibitem{1}\label{1} Hans Meissen en Rob van Oord, \textit{Schuiven met auto's, munten en bollen}, Epsilon Uitgaven, Utrecht, 2001-2002.
\bibitem{2}\label{2} Hans Lauwerier, \textit{Fractals, Meetkundige figuren in eindeloze herhaling}, Aramith Uitgevers, Bloemendaal, 1987.
\bibitem{3}\label{3} Miodrag S. Petkovi, \textit{Famous puzzles of great mathematicians}, American Mathematical Society, 2009.
\bibitem{4}\label{4} Keith Devlin, \textit{Wiskunde: Wetenschap van patronen en structuren}, Het Spectrum, Utrecht, 1998.
\bibitem{5}\label{5} Marcus du Sautoy, \textit{The Story of Maths}, BBC reeks, 2008.
\bibitem{6}\label{6} John J. O'Connor, Edmund F. Robertson, \textit{The MacTutor History of Mathematics archive}, \url{http://www-history.mcs.st-andrews.ac.uk/}
\bibitem{7}\label{7} Stuart Anderson, \textit{Squaring.Net}, \url{http://www.squaring.net/}
\bibitem{8}\label{8} Mike Askew, \textit{Meetkunde. Van $\pi$ tot Pythagoras}, Librero, 2011.
\bibitem{9}\label{9} Govert Schilling, \textit{Sterrenkunde van A tot Z}, Het Spectrum, Utrecht, 1999.
\bibitem{10}\label{10} Simon Singh, \textit{Het laatste raadsel van Fermat}, BV Uitgeverij De Arbeiderspers, Amsterdam, 2001.
\bibitem{11}\label{11} Ian Stewart, \textit{Over sneeuwkristallen en zebrastrepen}, Uitgeverij Uniepers Davidsfonds/Leuven Natuur en Techniek, Amsterdam, 2002.
\bibitem{12}\label{12} Gieljan de Vries, \textit{Computer stapelt sinaasappels}, \url{http://www.kennislink.nl/web/show?id=98583 - 21k}
\bibitem{13}\label{13} \textit{Sphere packing, internet}, \url{http://mathworld.wolfram.com/SpherePacking.html}
\bibitem{14}\label{14} \textit{Mathematical mysteries: Kepler's conjecture}, \url{http://plus.maths.org/issue3/xfile/index.html}
\bibitem{15}\label{15} \textit{The Kepler Conjecture}, University of Pittsburgh, \url{http://www.math.pitt.edu/~thales/kepler98/}
\bibitem{16}\label{16} Thomas Hales, \textit{Cannonballs and Honeycomb}, University of Pittsburgh \url{http://www.math.pitt.edu/articles/cannonOverview.html#honey}
\bibitem{17}\label{17} \textit{Kissing numbers}, \url{http://en.wikipedia.org/wiki/Kissing_number_problem}

\end{thebibliography}