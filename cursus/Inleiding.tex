\section{Inleiding}

\subsection{Kennismaking}


Doel is kennismaken met het vullen van vlakken en volumes.

Kennismaken met woorden en begrippen als:

\begin{enumerate}
	\item Vlakvulling:
	
	\begin{itemize}
	\item Identieke figuren
	\item De figuren mogen niet overlappen
	\item Het is niet steeds mogelijk om het vlak volledig te vullen: \\
	Zo effici�nt mogelijk $\rightarrow$ komen tot optimale vlakvulling $\rightarrow$ begrip effici�ntie
  \item (soms wel mogelijk: ) Complete vlakvulling	
  \item \textbf{BELANGRIJK!!!} Vullen van oneindig vlak (op tafel): Complete vlakvulling met regelmatige veelhoeken kan enkel bij 3, 4 en 6-hoeken.  (zie p. 74-75 op de gekopieerde papieren van Jordy) en zie ook jullie werk van didactiek wetenschap!!!!
  \item Vullen van een figuur (dienblad) 
  \item Kissing number? (zie bijvoorbeeld FPoGM p . 136 en zo, ook bestand van Jordy, mss eerder gewoon verwerken in andere les)
  \item Interessante site mss: http://users.skynet.be/fa071143/index.html
  
  
  \item[ ] Hoe kunnen we dit PRAKTISCH DOEN:
  
  \item Een soort 'discussieronde' om tot de 'defintie' van het begrip 'vlakvulling' te komen. De discussieronde kan op gang gebracht worden door enkele \textbf{belangrijke vragen}, door het \textbf{tonen van vlakvulling} (of juist niet), door het tonen van schilderijen (Escher bijvoorbeeld enzovoort, link met kunst leggen!!), door het tonenen van praktische problemen (het leggen van kaarten op een tafel ...). Ook zouden we de leerlingen enkele \textbf{praktische probleempjes} al kunnen aanbieden, dat ze zelf experimenteel kunnen uittesten.
  \item We kunnen hen ook een blaadje geven met iets 'moeilijkere' vragen waarbij ze eerst intu�tief iets mogen opschrijven zonder te testen en daarna via experimenten waarbij je de leerlingen materiaal ter beschikking stelt.
(Net zoals bij 'Schaduwen' van Gommaar). \\
TWEE DIMENSIONAAL
\begin{itemize}
	\item Is het steeds mogelijk om kleine vierkantjes te vinden die een gegeven vierkant opvullen?
	\item Is het steeds mogelijk om kleine driehoekjes te vinden die een gegeven driehoek opvullen?
  \item Is het steeds mogelijk om kleine cirkeltjes te vinden die een gegeven cirkel opvullen?
  \item Is het mogelijk om 'deze figuur' (een bepaalde figuur wordt getoond) te verkrijgen met volgende stukken (principe van tangram)
  
  \item[ ]DRIE DIMENSIONAAL
  \item	Is het steeds mogelijk om een kubus te vinden waar allemaal kleine kubusjes inpassen?
  \item Is het steeds mogelijk om een (regelmatige? Vierkante? Driehoekige?) piramide te vinden waar allemaal kleine piramides inpassen?
  \item Is het steeds mogelijk om een bol te vinden waar allemaal kleine bolletjes in passen?

\end{itemize}


  \item Hetzelfde kan eigelijk gedaan worden met het onderwerp hieronder.
  \item Mss daarna overgaan naar het begrip reguliere betegeling en de leerlingen zelf laten onderzoeken welke reguliere betegelingen er mogelijk zijn in het vlak via het geven van allemaal driehoekjes, vierkantjes, vijfhoeken, zeshoeken, achthoeken ... (zie dus  onderzoeksvraag van PP en JV van didactiek wetenschap
  
\end{itemize}

  \item Stapelproblemen: 
  
  \begin{itemize}
	\item identieke voorwerpen
	\item Het is niet steeds mogelijk om een volume volledig te vullen:
	
	\begin{itemize}
	  \item Zo effici�nt mogelijk $\rightarrow$ komen tot optimale vlakvulling
	  \item (soms wel mogelijk: ) Complete vlakvulling
  \end{itemize}
  \item Zowel in 2D (zie ook vlakvulling) als in 3D
  \item Bolstapelprobleem
\end{itemize}
  

\end{enumerate}



Het is zeker de bedoeling dat de leerlingen volgende zaken ontdekt hebben, zodanig dat we dit in de lessen erna nog kunnen gebruiken:

\begin{itemize}
	\item We kunnen een vlak enkel betegelen met driehoeken, vierhoeken en zeshoeken
	\item Alle belangrijke definities!!!
\end{itemize}

\subsection{Complete vlakvulling met regelmatige veelvlakken / Regelmatige tegelpatronen}

De leerlingen zelf laten ontdekken dat dit enkel kan bij 3, 4 en 6-hoeken, door hen bijvoorbeeld eerst in karton allemaal 3, 4,5,6 en 8-hoeken te geven en het is de bedoeling dat ze zelf gaan onderzoeken of je met deze veelhoeken een volledige bedekking krijgt (dit is dus niet hetzelfde als zeggen dat er een volledige vlakvulling is en dan uiteindelijk zeggen dat het een strikvraag is h�!).
Ik denk dat ze wel snel zullen aanvullen dat het niet lukt met 5- en 8-hoeken, maar het is natuurlijk belangrijk waarom. 
Zie pagina 75 voor de uitleg (in boek met papieren die Jordy heeft gekopieerd). 
Het is dus zo dat de grootte van de hoek van een regelmatige n-hoek 180� - 360�/n is, en dat dat dit een deler moet zijn van 360� om een volledige vlakvulling te kunnen hebben. Als je dat wat uitwerkt kom je er op dat dit hetzelfde is als zeggen dat 2n moet deelbaar zijn door n-2. Als je effectief n= 3, 4, ... invult, dan merk je idd dat dit klopt, maar ik zie niet in hoe je het zo kan doen.
Wat wel lukt is dat je kijkt en vertrekt van de 360� graden en dat je dit deelt door de natuurlijke getallen, namelijk: 360� /2= 180� ($\rightarrow$ komt niet overeen met de hoek van een veelhoek (eigenlijk dus van een tweehoek :p), 360�/3= 120� ($\rightarrow$ dit is de hoek van een regelmatige 6-hoek), 360�/4= 90� ($\rightarrow$ dit is de hoek van een regelmatige 4-hoek), 360�/5= 72� ($\rightarrow$ komt niet overeen met een veelhoek), 360�/6= 60� ($\rightarrow$komt overeen met regelmatige driehoek), verder moeten we niet gaan, want dan krijg je geen zowiezo geen hoeken meer van regelmatige veelhoeken.

In dit stuk moet dus hetzelfde komen als in het werk didactiek wetenschappen van PP en JV


Daarna kunnen we een beetje de link leggen met Wiskunde en Kunst en andere tegelpatronen en tekeningen van Escher...

\subsection{Vullen van vlakken met driehoeken, vierkanten en zeshoeken}

We hebben gezien dat je een vlak enkel volledig kan vullen met driehoeken, vierkanten en zeshoeken. 

Een voorbeeld van driehoeken in driehoeken is de driehoek van serpinski.

In de volgende les zullen we ons bezighouden met het vullen van vierkanten met vierkanten.

Het betegelen van een vlak met zeshoeken zal de leidraad zijn bij het de stapelproblemen in het vlak.

In de eerste lessen zullen we het hebben over compete vullingen (les over vierkanten vullen met vierkanten en perfecte vierkanten en de les over de meetkundige interpretaties van kwadratische vergelijkingen en kubische vergelijkingen).