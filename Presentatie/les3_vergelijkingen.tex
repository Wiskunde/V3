\begin{frame}
  \frametitle{Les 3: Vergelijkingen op lossen met oppervlakten}
  \framesubtitle{Voorbeeld: $x^2+4x-5=0$}
  $$
  x^2+4x-5=0
  $$
  {\bf Algebra: } \visible<2->{$Opl=\{-5,1\}$.}\\
  {\bf Meetkunde: }\\   
  \begin{columns}
    \begin{column}{0.4\textwidth}
    \begin{tikzpicture}[scale=0.5, line cap=round,line join=round,>=triangle 45,x=1.0cm,y=1.0cm]
\clip(-1,-1) rectangle (8,8);
\filldraw[line width=1.6pt,fill=black,fill opacity=0.1] (0,0) -- (5,0) -- (5,5) -- (0,5) -- cycle;
\filldraw[line width=1.6pt,fill=black,fill opacity=0.1] (5,0) -- (7,0) -- (7,5) -- (5,5) -- cycle;
\filldraw[line width=1.6pt,fill=black,fill opacity=0.1] (0,5) -- (5,5) -- (5,7) -- (0,7) -- cycle;
\filldraw[line width=1.6pt,fill=black,fill opacity=0.1] (5,5) -- (7,5) -- (7,7) -- (5,7) -- cycle;
\draw (2.56,0) node[anchor=north] {$x$};
\draw (-0.5,2.87) node[anchor=north] {$x$};
\draw (6,0) node[anchor=north] {$2$};
\draw (-0.5,6.5) node[anchor=north] {$2$};
\end{tikzpicture}

    \end{column}
    \begin{column}{0.6\textwidth}
      \small
      \begin{itemize}
        \item \visible<3->{$x^2$ zien als een vierkant met zijde $x$}
        \item \visible<5->{$4x$ zien als twee rechthoeken}
        \item \visible<7->{Oppervlakte is gelijk aan $5$}
        \item \visible<8->{De figuur vervolledigen}
        \item \visible<10->{Nieuwe oppervlakte is $5+4=9$}
        \item \visible<11->{Zijde is $x+2=+\sqrt{9}$}
      \end{itemize}
      \vspace*{0.75cm}
    \end{column}
  \end{columns}  
\end{frame}
