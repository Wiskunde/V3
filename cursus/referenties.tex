%\section{Referenties}

\begin{thebibliography}{8}
\bibitem{1}\label{1} Hans Meissen en Rob van Oord, \textit{Schuiven met auto's, munten en bollen}, Epsilon Uitgaven, Utrecht, 2001-2002.
\bibitem{2}\label{2} Hans Lauwerier, \textit{Fractals, Meetkundige figuren in eindeloze herhaling}, Aramith Uitgevers, Bloemendaal, 1987.
\bibitem{3}\label{3} Miodrag S. Petkovi, \textit{Famous puzzles of great mathematicians}, American Mathematical Society, 2009.
\bibitem{4}\label{4} Keith Devlin, \textit{Wiskunde: Wetenschap van patronen en structuren}, Het Spectrum, 1998.
\bibitem{5}\label{5} Marcus du Sautoy, \textit{The Story of Maths}, BBC reeks, 2008.
\bibitem{6}\label{6} John J. O'Connor, Edmund F. Robertson, \textit{The MacTutor History of Mathematics archive}, \url{http://www-history.mcs.st-andrews.ac.uk/}
\bibitem{7}\label{7} Stuart Anderson, \textit{Squaring.Net}, \url{http://www.squaring.net/}
\bibitem{8}\label{8} Mike Askew, \textit{Meetkunde. Van $\pi$ tot Pythagoras}, Librero, 2011.
\bibitem{9}\label{9} http://www.luek.nl/illusie\_puzzle\_driehoek.php
\end{thebibliography}
