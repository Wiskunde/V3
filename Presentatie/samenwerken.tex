\begin{frame}
  \frametitle{Samenwerken}
%  \framesubtitle{}
  Website: \url{http://wiskunde.github.com/V3/}\\
  \vspace*{0.5cm}
  \begin{columns}[t]
    \begin{column}{0.5\textwidth}
      Samenwerking gebaseerd op:
      \begin{itemize}
        \item Overleg\\
          \visible<2->{
          \begin{itemize}
            \item Vergaderen
            \item Email
          \end{itemize}
          }
        \item \LaTeX\\
          \visible<3->{
          \begin{itemize}
            \item Fantastische typesetting!
            \item Simpele tekstbestanden
            \item \href{http://www.tug.org/twg/mactex/tutorials/ltxprimer-1.0.pdf}{Boek}
          \end{itemize}
          }
        \item Git\\
          \visible<4->{
          \begin{itemize}
            \item Elk heeft eigen versie
            \item Samen versie online
            \item \href{http://git-scm.com/book}{Boek}
          \end{itemize}
          }
      \end{itemize}
    \end{column}
    \begin{column}{0.5\textwidth}
      \visible<5->{
        Voordelen voor leerkrachten:
        \begin{itemize}
          \item Eigen versie:
          \visible<6->{
            \begin{itemize}
              \item Huisstijl
              \item Herschikken materiaal
              \item Materiaal naar keuze
              \item Inpassen in eigen lessen
            \end{itemize}
          }
          \item Versie online:
          \visible<7->{
            \begin{itemize}
              \item Verbeteren materiaal
              \item Toevoegen materiaal
              \item Didactiek verbeteren
            \end{itemize}
          }
        \end{itemize}
      }
    \end{column}
  \end{columns}
\end{frame}
