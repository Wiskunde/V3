
\begin{frame}
\frametitle{Driedimensionale stapelproblemen}
\framesubtitle{Kanonskogels stapelen: combinaties van stapels}

\begin{block}{Vraag:Is nu mogelijk om van een driehoekige stapel van een bepaalde hoogte een vierkante piramide te maken, zonder dat je kogels overhoudt?}
Neen, zie Frist Beukers en Jaap Top (1982).
\end{block}

\begin{block}{Vraag: Kan je van twee stapels knikkers in driehoekige piramides een nieuwe vierkante piramide maken?}
Ja, want \[d(n-1)-d(n)=v(n).\]
\end{block}

\begin{block}{Vraag: Kan je van een combinatie van een driehoekige en vierkante piramides een andere combinatie van een driehoekige en vierkante piramide maken?}
Ja, want \[ v(n)+d(n+1)= d(n-1) + v(n+1).\]
\end{block}
\end{frame}

