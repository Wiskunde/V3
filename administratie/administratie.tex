\documentclass[12pt]{article}

\textwidth 16cm \textheight 23cm \evensidemargin 0cm
\oddsidemargin 0cm \topmargin -2cm
\parindent 0pt
\parskip \medskipamount


\usepackage[dutch]{babel}
\usepackage{amssymb}
\usepackage{amsmath}
\usepackage[utf8,latin1]{inputenc}
\usepackage[normalem]{ulem} % strikethrough normal text with \sout{text}
\usepackage{cancel} % strikethrough in math mode with \cancel{text}
%\usepackage{gensymb}

\usepackage{subfig}
\usepackage{wrapfig}
\usepackage{graphicx}
\usepackage{graphics}
\usepackage{latexsym}
%\usepackage[latin1]{inputenc} %letters met accenten mogen in broncode gebruikt worden
\usepackage{amssymb,amsmath,amsthm} %allerlei wiskundige symbolen => http://www.ams.org/tex/amslatex.html voor read-me's
%\usepackage[small,bf,hang]{caption}
\usepackage[table]{xcolor}
\usepackage{pgf,tikz}
\usetikzlibrary{arrows}

\usepackage[pdftex]{hyperref} 
\hypersetup{setpagesize=false}
%\hypersetup{bookmarksopen}
\hypersetup{bookmarksnumbered} %voor de bookmarks in pdf de nrs zetten ook
\hypersetup{pdfborder=0.1pt} %de rand rond de links smaller maken
\hypersetup{linktocpage=true} %titels of pgnrs omkaderen in toc
\hypersetup{colorlinks=true} %kaders rond links of tekst in kleur, kaders gaan weg bij afprinten, kleur blijft


\numberwithin{equation}{section} %om de equations labels te geven die met section meegaan, dus niet gwn (1) (2) ... maar (1.1) (1.2) ...

%iets voor modulonotatie
\makeatletter
\def\imod#1{\allowbreak\mkern10mu({\operator@font mod}\,\,#1)}
\makeatother

%Getallenverzamelingen
\newcommand{\N}{\mathbb{N}}
\newcommand{\Z}{\mathbb{Z}}
\newcommand{\Q}{\mathbb{Q}}
\newcommand{\R}{\mathbb{R}}
\newcommand{\C}{\mathbb{C}}
\newcommand{\mO}{\mathcal{O}}
%<= en >= mooier maken
\renewcommand{\leq}{\leqslant}
\renewcommand{\geq}{\geqslant}

%speciale wiskunde letters: \mathcal{} of \mathfrak{}

%Nieuwe omgevingen voor definities en stellingen
\theoremstyle{plain}  \newtheorem{stel}{Stelling}[section]
\theoremstyle{plain}  \newtheorem{lemma}[stel]{Lemma}
\theoremstyle{plain}  \newtheorem{gevolg}[stel]{Gevolg}
\theoremstyle{definition}  \newtheorem{defi}[stel]{Definitie}
\theoremstyle{definition} \newtheorem{vb}[stel]{Voorbeeld}
\theoremstyle{definition} \newtheorem{vbn}[stel]{Voorbeelden}
\theoremstyle{definition}  \newtheorem{opm}[stel]{Opmerking}
\theoremstyle{plain}  \newtheorem{opdracht}{Opdracht}

\hyphenation{e-ner-zijds}
\hyphenation{ka-nons-ko-gel}
\hyphenation{ver-ge-lij-king}
\hyphenation{schrij-ven}
\hyphenation{ka-nons-ko-gels}
\hyphenation{ont-stond}
\hyphenation{kwa-dra-tisch}
\hyphenation{ver-ge-lij-kin-gen}

\usepackage{color}
\newcommand{\todo}[1]{\textcolor{red}{\##1\#}}
\newcommand{\question}[1]{\textcolor{blue}{\##1\#}}

% Wissel tussen versie van leerkracht en versie voor leerlingen
% gebruik: \teacher{Dit is een notitie voor de leerkracht}
% eerste parameter is de notitie
\newcommand{\teacher}[1]{\textcolor{gray}{\emph{#1}}}
%\newcommand{\teacher}[1]{}

% Wissel tussen versie van leerkracht en versie voor leerling door antwoorden te verbergen
% gebruik: \answer[2cm]{Dit is een antwoord}
% tussen vierkante haakjes is optioneel de lengte die voorzien moet worden (default = 0cm),
% de parameter is het antwoord
\newcommand{\answer}[2][0cm]{{\textcolor{gray}{\emph{#2}}}}
%\newcommand{\answer}[2][0cm]{\vspace*{#1}}

\graphicspath{{../cursus/figuren/}}

%\newtheorem{definitie}{Definitie}

\newcommand{\degree}{\ensuremath{^\circ}}

\begin{document}

\thispagestyle{empty}

\begin{center}
 \begin{minipage}{7cm}
\begin{center}
%\aula 
%B


\includegraphics[width=5cm]{ruglogo.pdf}


 {\large
Universiteit Gent \\
Faculteit  Wetenschappen \\
Vakgroep  Wiskunde
} \end{center}\end{minipage}

\vspace{3cm}

 {\Huge\bf 
  Vullen van Vlakken en Volumes:\\ Administratie}
  
\baselineskip=12pt
\vspace{1.5cm}

{\large\bf  Lien Lambert }\\
\vspace{0.5cm}
{\large \bf Pieter Pareit }\\
\vspace{0.5cm}
{\large \bf Jordy Vanpoucke}\\

\baselineskip=12pt

\vspace{2cm}

 
\bigskip

{\large Academiejaar 2011-2012}

%\vspace{5cm}
%\baselineskip=1.2cm



\end{center}

\vfill

\hspace*{\fill}
\begin{minipage}{8cm}
\noindent   Project Vakdidactiek Wiskunde II 
 \end{minipage}


\bigskip

\hspace*{\fill}
\begin{minipage}{8cm}
\noindent  Prof. Dr.  F. De Clerck\\
Praktijkassistent G. Maes
 \end{minipage}

%\newpage
%\tableofcontents 
%\newpage

\section{Doelgroep}
Leerlingen uit de tweede graad ASO. 

Uiteraard is dit pakket dan ook toegankelijk voor leerlingen uit de derde graad, maar misschien is dit voor bepaalde richtingen dan te eenvoudig of niet uitdagend genoeg. Ook is er een betere link met de eindtermen voor de tweede graad, zodat dit pakket dan ook onmiddellijk aansluit bij bepaalde onderwerpen die de leerlingen sowieso krijgen.

\section{Voorkennis}
In dit lessenpakket over het vullen van vlakken en volumes steunen we expliciet op de geziene leerstof uit
\begin{itemize}
\item de getallenleer en algebra: De leerlingen
\begin{itemize}
\item (19) kunnen vergelijkingen van de eerste en de tweede graad in \'{e}\'{e}n onbekende oplossen,
\item (20) kunnen ongelijkheden van de eerste en de tweede graad in \'{e}\'{e}n onbekende oplossen,
\item (21) lossen problemen op die kunnen vertaald worden naar:
\begin{itemize}
\item een vergelijking van de eerste en de tweede graad in \'{e}\'{e}n onbekende.
\item een ongelijkheid van de eerste en de tweede graad in \'{e}\'{e}n onbekende.
\end{itemize}
\end{itemize}
\item de re\"{e}le functies: De leerlingen
\begin{itemize}
\item (28) kunnen stelsels van twee vergelijkingen van de eerste graad met twee onbekenden algebra�sch oplossen en de oplossing grafisch interpreteren,
\item (29) kunnen problemen oplossen die te vertalen zijn naar stelsels van twee vergelijkingen van de eerste graad met twee onbekenden,
\item (31) lossen problemen op die kunnen beschreven worden met eerste- en tweedegraadsfuncties.
\end{itemize}
\item de meetkunde: De leerlingen
\begin{itemize}
\item (34) verklaren gelijkvormigheid van figuren met behulp van schaal en congruentie,
\item (35) gebruiken de gelijkvormigheid van driehoeken en de stelling van Thales om de lengte van lijnstukken te berekenen,
\item (39) kunnen problemen met zijden en hoeken van driehoeken uit de technische wereld oplossen door een effici�nte keuze te maken uit:
\begin{itemize}
\item de stelling van Thales
\item de stelling van Pythagoras
\item goniometrische getallen
\end{itemize}
\item (41) lossen eenvoudige problemen i.v.m. ruimtelijke situaties op door gebruik te maken van eigenschappen van vlakke figuren,
\item (43) kunnen de inhoud van sommige ruimtelijke objecten benaderend berekenen door ze op te splitsen in of aan te vullen tot gekende lichamen,
\item (44) kunnen effecten van schaalverandering op inhoud en oppervlakte berekenen.
\end{itemize}
\end{itemize}

\section{Globale motivatie}
Voor dit lessenpakket zijn we vertrokken vanuit het bolstapelprobleem van Kepler. Het viel ons op hoe men in het dagelijkse leven bezig is met het zo optimaal mogelijk stapel van bollen, maar ook van kubussen en andere voorwerpen. In dit lessenpakket zullen we naar dit bolstapelprobleem toewerken door te starten met het tweedimensionale probleem, namelijk de vlakvulling. Hierbij is het de bedoeling dat de leerlingen zelf aan de slag gaan, individueel of in groep, en in staat zijn om zelf idee\"{e}n te formuleren, hypotheses stellen, waarbij ze zullen experimenteren om tot slot hun bevindingen ook hard maken via het bewijzen of uitrekenen ervan. Daarnaast is het ook de bedoeling om de leerlingen wat cultuur bij te brengen en hen in aanraking te laten komen met \textquoteleft wiskunst'.
\subsection{Vakoverschrijdende eindtermen}
In het lessenpakket komen de volgende vakoverschrijdende eindtermen aan bod.
\subsubsection{Gemeenschappelijke stam}
De leerlingen	 
\begin{itemize}
\item	(2) kunnen originele idee\"{e}n en oplossingen ontwikkelen en uitvoeren,
\item (4)	blijven, ondanks moeilijkheden, een doel nastreven,
\item (6)	kunnen schoonheid ervaren, 
\item (7)	kunnen schoonheid cre\"{e}ren,
\item (8)	benutten leerkansen in diverse situaties,
\item (10) engageren zich spontaan,
\item (11) kunnen gegevens, handelwijzen en redeneringen ter discussie stellen a.d.h. van relevante criteria,
\item (12) zijn bekwaam om alternatieven af te wegen en een bewuste keuze te maken,
\item (13) kunnen onderwerpen benaderen vanuit verschillende invalshoeken,
\item (19) dragen actief bij tot het realiseren van gemeenschappelijke doelen.
\end{itemize}
\subsubsection{Leren leren}
De leerlingen
\begin{itemize}
\item (4)	kunnen verwerkte informatie vakoverstijgend en in verschillende situaties functioneel toepassen,
\item (5)	kunnen informatie samenvatten,
\item (6)	kunnen op basis van hypothesen en verwachtingen mogelijke oplossingswijzen realistisch inschatten en uitvoeren,
\item (7)	evalueren de gekozen oplossingswijze en de oplossing en gaan eventueel op zoek naar een alternatief.
\end{itemize}
\subsubsection{Contexten}
We maken enkel gebruik van eindtermen uit context 7, `socioculturele samenleving'.\\
De leerlingen
\begin{itemize}
\item (6) gaan actief om met de cultuur en kunst die hen omringen,
\item (7)	illustreren de wederzijdse be�nvloeding van kunst, cultuur en techniek, politiek, economie, wetenschappen en levensbeschouwing.
\end{itemize}
\subsection{Vakgebonden eindtermen}
We werken toe naar volgende vakgebonden eindtermen.\\
De leerlingen
\begin{itemize}
\item (1)	begrijpen en gebruiken wiskundetaal,
\item (2)	passen probleemoplossende vaardigheden toe,
\item (3)	verantwoorden de gemaakte keuzes voor representatie- en oplossingstechnieken,
\item (4)	controleren de resultaten op hun betrouwbaarheid,
\item (5)	gebruiken informatie- en communicatietechnologie om wiskundige informatie te verwerken, berekeningen uit te voeren of wiskundige problemen te onderzoeken,
\item (6) gebruiken kennis, inzicht en vaardigheden die ze verwerven in wiskunde bij het verkennen, vertolken en verklaren van problemen uit de realiteit,
\item (7)	kunnen voorbeelden geven van re\"{e}le problemen die m.b.v. wiskunde kunnen worden opgelost, 
\item (8)	kunnen voorbeelden geven van de rol van de wiskunde in de kunst.
\end{itemize}
De leerlingen
\begin{itemize}
\item (9*) ervaren het belang en de noodzaak van bewijsvoering, eigen aan de wiskunde,
\item (10*)	ervaren dat gegevens uit een probleemstelling toegankelijker worden door ze doelmatig weer te geven in een geschikte wiskundige representatie of model, 
\item (11*)	ontwikkelen zelfregulatie: het ori�nteren op de probleemstelling, het plannen, het uitvoeren en het bewaken van het oplossingsproces,
\item (12*)	ontwikkelen zelfvertrouwen door succeservaring bij het oplossen van wiskundige problemen,
\item (13*)	ontwikkelen bij het aanpakken van problemen zelfstandigheid en doorzettingsvermogen,
\item (14*) werken samen met anderen om de eigen mogelijkheden te vergroten.
\end{itemize}

Hoewel we in zekere zin uitgingen van de onderstaande vakgebonden eindtermen als voorkennis, zal deze lessenreeks er ook voor zorgen dat de leerlingen onderstaande vaardigheden en kennis nog verder ontwikkelen:


\begin{itemize}
	\item (19) kunnen vergelijkingen van de eerste en de tweede graad in \'{e}\'{e}n
	\item (39) kunnen problemen met zijden en hoeken van driehoeken uit de technische wereld oplossen door een effici�nte keuze te maken uit:
\begin{itemize}
\item de stelling van Thales
\item de stelling van Pythagoras
\item goniometrische getallen
\end{itemize}
  \item (41) lossen eenvoudige problemen i.v.m. ruimtelijke situaties op door gebruik te maken van eigenschappen van vlakke figuren,


\end{itemize}



\section{Planning, lesdoelstellingen, werkvormen en didactische wenken}
\subsection{Les 1: Inleiding}
\subsubsection{Lesdoelstellingen}
\begin{itemize}
\item De leerlingen kennen de definities van een vlakvulling, een complete vlakvulling en een optimale vlakvullling.
\item De leerlingen kunnen in groep komen tot bevindingen over het compleet vullen van een oneindig groot vlak.
\item De leerlingen kunnen bewijzen dat een betegeling van een oneindig groot vlak met regelmatige veelhoeken enkel lukt met regelmatige driehoeken, regelmatige vierhoeken en regelmatige zeshoeken.
\item De leerlingen kennen stapelproblemen.
\item De leerlingen kunnen zelfstandig op zoek gaan naar meer informatie over het lesonderwerp.
\item De leerlingen kunnen in groep samenwerken om enkele raadsels op te lossen.
\end{itemize}

\subsubsection{Werkvormen}
\begin{itemize}
  \item Onderwijsleergesprek.
  \item Groepswerk.
  \item Leerlingen laten puzzelen en experimenteren.
\end{itemize}

\subsubsection{Didactische wenken}
In de eerste les is het de bedoeling om samen met de leerlingen te komen tot een definitie van een vlakvulling, waarbij wordt gekeken in hoeverre het mogelijk is om een vlak compleet te vullen. Zo komen de leerlingen dan ook tot de definitie van een optimale vlakvulling, een complete vlakvulling en effici\"{e}ntie.

Daarnaast worden de leerlingen aangemoedigd om in groep op zoek te gaan naar oplossingen voor de gestelde problemen, waarbij ze al experimenterend tot hun bevindingen komen. Daarbij is het aan te raden om als leerkracht rond te kijken en de leerlingen eventueel bij te sturen indien dit nodig zou zijn. Dit stuk vraagt wel wat voorbereiding om alle figuren op voorhand uit karton te knippen.

Vervolgens komt het begrip stapelprobleem aan bod om zo de belangrijkste begrippen al eens aangekaart te hebben tijdens de eerste les.

Tot slot kunnen de leerlingen zelf op onderzoek gaan naar meer informatie over het vullen van vlakken en volumes en worden ze uitgedaagd om enkele puzzels en/of raadsels op te lossen. Dit om hen te prikkelen en nieuwsgierig te maken naar de volgende lessen.

\subsection{Les 2: Vullen van vierkanten met kleinere vierkanten en zo tot Perfecte Vierkanten komen}
\subsubsection{Lesdoelstellingen}

\begin{itemize}
  \item De leerlingen kunnen een een vierkant met kleinere vierkanten vullen van de zelfde grootte, om zo een algemene formule die de maximale effici�ntie bepaald, te vinden.
  \item De leerlingen weten wat een perfect vierkant van orde $n$ is.
  \item De leerlingen kunnen een ondergrens zoeken voor de orde van het perfect vierkant door redeneren.
  \item De leerlingen zien in dat een perfect vierkant dat gedraaid of gespiegeld wordt geen ander perfect vierkant oplevert. Hiervoor leren de leerlingen het begrip isomorfisme.
  \item De leerlingen beseffen dat het vinden van ��n perfect vierkant leidt tot het vinden van een oneindig aantal perfecte vierkanten. 
  \item De leerlingen kunnen een bewijs uit het ongerijmde opstellen om een eigenschap te bewijzen over perfecte vierkanten.
  \item De leerlingen kunnen hun resultaten en definities in verband met perfecte vierkanten uitbreiden naar ��n en drie dimensies.
\end{itemize}

\subsubsection{Werkvormen}

\begin{itemize}
  \item Onderwijsleergesprek.
  \item Individueel werken.
  \item Leerlingen laten puzzelen.
\end{itemize}

\subsubsection{Didactische wenken}

Het is de bedoeling dat de leerkracht samen met de leerlingen een theorie rond perfecte vierkanten ontwikkelt.

We beginnen samen met de leerlingen aan het vullen van een vierkant met gelijke vierkanten. Hiervoor berekenen we dan de effici�ntie, dus hoeveel van het vierkant wordt bedekt door kleinere vierkanten. Hiervoor kunnen we snel een algemene formule vinden.

Dan wensen we het probleem moeilijker maken. Hiervoor eisen we dat een vierkant volledig gevuld wordt met kleinere vierkanten die allemaal een verschillende grootte hebben. Omdat dit een moeilijk probleem is, bijvoorbeeld naar het perfect vierkant met de laagste orde is 60 jaar gezocht, geven we in deze cursus zelf een voorbeeld en een beetje geschiedenis, wat de leerkracht aanbrengt.

De leerlingen die toch zelf actief aan de slag willen gaan, en een perfect vierkant puzzelen, geven we deze kans. Voor hen werden in de appendix de bouwplannen voorzien voor alle perfecte vierkanten van orde 22. Een perfect vierkant vinden is heel moeilijk, maar wanneer je reeds weet wat de groottes zijn van de kleinere vierkantjes wordt het toch haalbaar.

Voorgaande is optioneel en kan gerust door de leerlingen die graag puzzelen thuis gedaan worden. Een competitie is ook mogelijk. Hierna gaan we door. We kunnen aantonen dat de orde minimaal een bepaalde grootte moet zijn. Voor een lage orde lukt dit. We nemen daarna ��n bepaald perfect vierkant en tonen dan aan dat vanuit dit perfecte vierkant nog een ander perfect vierkant geconstrueerd kan worden.

Als laatste bewijzen we een bepaalde eigenschap over een algemeen perfect vierkant. Deze eigenschap wordt dan gebruik om aan te tonen dat een perfect vierkant niet uitgebreid kan worden tot een perfecte kubus. Deze eigenschappen worden bewezen uit het ongerijmde, maar kunnen heel visueel aangepakt worden. Hiervoor moeten dan wel een paar vierkanten van verschillende grootte uit een stuk karton geknipt worden. 


\subsection{Les 3: Puzzelen met vierkanten en rechthoeken om vergelijkingen op te lossen}

\subsubsection{Lesdoelstellingen}

\begin{itemize}
  \item De leerlingen kunnen voor hen gekende methoden, uit de algebra, gebruiken om problemen op te lossen, i.h.b. de methode voor het oplossen van een vierkantsvergelijking.
  \item De leerlingen kunnen dan dezelfde problemen op een meetkundige manier oplossen.
  \item De leerlingen beseffen dat er een link is tussen algebra en meetkunde.
  \item De leerlingen hebben een idee hoe het oplossen van derdegraadsvergelijkingen historisch is gegroeid.
  
\end{itemize}

\subsubsection{Werkvormen}

\begin{itemize}
  \item Onderwijsleergesprek.
  \item Individueel werken.
  \item Videofragment.
\end{itemize}

\subsubsection{Didactische wenken}

Het hoofddoel van deze les is om de link tussen algebra en meetkunde bij het oplossen van vergelijkingen aan de leerlingen mee te geven.

De leerlingen zouden individueel de vraagjes moeten kunnen oplossen. De leerkracht kan de leerlingen bijstaan door samen met hen de gebruikte vierkanten te teken en waar nodig de lengtes van de zijden aan duiden. Een moeilijkheid voor de leerlingen is het gegeven dat ze het vierkant $x^2$ moeten schetsen zonder dat ze echt een waarde voor $x$ kennen. Het kan goed zijn om het bord in twee te verdelen en op het linker deel een schets maken met een \textquoteleft grote' $x$ en op het linker bord een schets maken met een \textquoteleft kleine' $x$. Verdere stappen zouden dan toch tot het zelfde resultaat moeten leiden.

De op te lossen vergelijkingen zijn nooit moeilijk. De methode om ze met het vullen van vierkanten en rechthoeken op te lossen wordt wel steeds moeilijker. Niet alle vergelijkingen kunnen met de methode opgelost worden. Zo kunnen we voor $ax^2 + bx + c = 0$ met $a>0$ en $c>0$ geen methode meer vinden omdat $ax^2+bx$ niet meer gelijk gesteld kan worden aan een oppervlakte. Hier moet wel rekening mee gehouden worden in de vraagstelling of indien er extra oefeningen gegeven worden.

Het verhaaltje dat bij het oplossen van derdegraadsvergelijkingen hoort komt uit de BBC serie: 'The Story of Maths' door Marcus du Sautoy. Deze reeks kan in de stadsbibliotheek van Gent gevonden worden (Siso: 510.2). Daarnaast kan deze bekeken worden op youtube. Het relevant videofragment staat op aflevering 2: 'The genius of the east' (\url{http://www.youtube.com/watch?v=9mz7M8ed_hE}) en loopt van 51:45 tot 55:45.

\subsection{Les 4: Kennismaking met tweedimensionale stapelproblemen en hiermee experimenteren}

\subsubsection{Lesdoelstellingen}
\begin{itemize}
\item De leerlingen beseffen dat je een tafel niet volledig kan bedekken met cirkels
en dat we daarom op zoek kunnen gaan naar de optimale configuratie, namelijk 
het hexagonaal rooster.

\item De leerlingen weten wat een voronoicel is en kunnen hiermee de effici�ntie
 van een rooster berekenen.

\item De leerlingen kunnen de oppervlakte van een driehoek, rechthoek, zeshoek en cirkel
berekenen.

\item De leerlingen kunnen a.d.h.v. een geogebra-applet of door te experimenteren met jetons of munten de effici�ntie van dienbladen bepalen.

\item De leerlingen kunnen de effici�ntie voor eenvoudige voorbeelden uitrekenen.

\item De leerlingen ontwikkelen meetkundig inzicht.

\end{itemize}

\subsubsection{Werkvormen}
\begin{itemize}
\item Onderwijsleergesprek.
\item Groepswerk.
\item Leerlingen laten experimenteren.
\end{itemize}
\subsubsection{Didactische wenken}
Op weg naar het bolstapelprobleem van Kepler houden we eerst halt bij het tweedimensionale equivalent. Hiervoor trekken we drie lessen uit.

In het eerste deel van de eerste les van \textquoteleft tweedimensionale stapelproblemen' werken de leerlingen samen met de leerkracht aan de hand van de cursus. Het is aangewezen dat de leerlingen al in kleine groepjes samen zitten, waarbij de leerkracht echter eerst nog vooraan in de klas de leider is van het onderwijsleergesprek. In de cursus staan vragen en opdrachten. Het is de bedoeling dat de leerkracht de leerlingen aanzet om hierover na te denken of effectief de opdracht uit te voeren. Dit kan gaan van een kleine berekening tot effectief experimenteren met munten. 

Als we bij het dienbladenprobleem zijn aanbeland is het effectief aan de leerlingen om aan de slag te gaan. Ze moeten voor verschillende aantallen munten en verschillende vormen van dienbladen (cirkelvormig, (regelmatige) driehoekig, (regelmatige) zeshoekig, rechthoekig) de effici�ntie berekenen. Het is de bedoeling dat ze dit experimenteel uitvoeren, a.d.h.v. Geogebra-applets of met behulp van jetons (of munten) en karton.
Het ideale zou zijn om de klas te verdelen in twee grote groepen waarvan een deel kan beginnen aan de computers, een ander deel kan starten met het werken met het andere materiaal. 
Het zou handig zijn om in het klaslokaal te zitten waarin bijvoorbeeld rondom computers staan, of laptops, maar waarbij de klas ook over gewone banken beschikt om les te volgen.
Indien dit niet ter beschikking is op de school kunnen we deze les en experimenteerfase in een PC-lokaal geven. Het experimenteren met de jetons is dan wel iets
moeilijker te combineren. Daarom is het in dat geval handig om dan de volledige klas eerst te laten werken met PC's en dan (eventueel in de tweede les) 
met het andere materiaal.

Het is echter ook de bedoeling om de leerlingen sommige configuraties ook echt te laten tekenen en hun de effici�ntie te laten berekenen.
De leerkracht kan de leerlingen zeggen dat ze dit voor een bepaald aantal moeten doen of voor enkele specifieke gevallen.
Je kan de klas ook opsplitsen, zodanig dat niet iedereen alle effici�nties moet berekenen, maar dat je met de ganse klas wel degelijk komt tot een volledig ingevulde tabel.


\subsection{Les 5: Het dienbladenprobleem: experimenteren en effectief berekenen}
\subsubsection{Lesdoelstellingen}

\begin{itemize}
\item De leerlingen kunnen a.d.h.v. een Geogebra-applet of door te experimenteren met jetons(of munten) de effici�ntie van dienbladen bepalen.
\item De leerlingen kunnen functioneel gebruik maken van ICT.
\item De leerlingen kunnen de effici�ntie voor eenvoudige voorbeelden uitrekenen.
\item De leerlingen kunnen vanuit een figuur meetkundig inzicht verwerven en hierdoor problemen vereenvoudigen.
\item De leerlingen kunnen de meetkundige interpretatie van de goniometrische formules functioneel gebruiken, i.h.b. voor projecties.
\end{itemize}

\subsubsection{Werkvormen}
\begin{itemize}
\item Onderwijsleergesprek.
\item Groepswerk.
\item Leerlingen laten experimenteren.
\end{itemize}

\subsubsection{Didactische wenken}

In deze les gaan de leerlingen verder met experimenteren en met het berekenen van de effici�ntie van sommige gevallen. De leerkracht blijft in de klas rondlopen en helpt waar nodig.
Als de klas was opgedeeld in een helft aan de computer en een helft met het andere materiaal, is het aangewezen om nu te wisselen.
In het tweede deel van deze les is het handig om samen met de leerlingen alle bekomen resultaten samen te voegen. Dan kunnen de effici�nties van de configuraties met 1, 2 of 3 munten worden besproken.


\subsection{Les 6: Het dienbladenprobleem: berekenen, bewijzen en besluiten}
\subsubsection{Lesdoelstellingen}
\begin{itemize}
\item De leerlingen beseffen dat men uit experimenteren belangrijke informatie kan halen en beseffen ook het belang van bewijzen.
\item De leerlingen kunnen aan de hand van een stappenplan het bewijs zelf verwerven.
\item De leerlingen kunnen de effici�ntie van ingewikkelde configuraties berekenen.
\item De leerlingen kunnen vorige berekeningen gebruiken om ingewikkelde configuraties te vereenvoudigen.
\item De leerlingen kunnen uit experimentele resultaten conclusies trekken over de meest effici�nte dienbladen, en kunnen ook (mogelijke) verklaringen geven

\end{itemize}



\subsubsection{Werkvormen}
\begin{itemize}
\item Onderwijsleergesprek.
\item Groepswerk.
\item Leerlingen laten experimenteren.
\end{itemize}

\subsubsection{Didactische wenken}
Deze les is vooral bedoeld om de leerlingen te doen inzien dat ze niet steeds zomaar op de intu�tie kunnen afgaan. Het uitproberen met Geogebra is leuk, het experimenteren met figuren ook. Je ziet dan snel in `op het gevoel' dat je niet effici�nter kan, maar het is belangrijk om dit ook aan te tonen. We zullen het geval behandelen van een driehoekig dienblad met vier munten. In dit geval is de intu�tie wel degelijk correct want we vinden er een bewijs voor. De leerlingen kunnen dit bewijs zelf doorlopen met deze cursus en dit begrijpen via een figuur die ze zelf knutselen.

Op het einde van de les kan de leerkracht nog eens terugkomen op enkele andere specifieke gevallen voor het berekenen van de configuratie.

Belangrijk is om op het einde de volledige tabel, die in de vorige les normaalgezien al was samengelegd, te overlopen en er de belangrijkste zaken uit te halen over wat nu de effici�ntste dienbladen zijn in welke gevallen dit klopt met de intu�tie. Ook leuk om op te merken is dat in de realiteit de dienbladen die in de horeca gebruikt worden net zulke dienbladen zijn waarvan we vinden dat ze het effici�ntst zijn!

\subsection{Les 7: Kanonskogels stapelen}

\subsubsection{Lesdoelstellingen}

\begin{itemize}
  \item De leerlingen weten wat het bolstapelprobleem inhoudt.
	\item De leerlingen kunnen het aantal kanonskogels in een (regelmatige) piramide met vierkant en driehoekig grondvlak met $n$ lagen experimenteel berekenen.
	\item De leerlingen kunnen vanuit een gegeven rij het recursief en/of expliciet voorschrift geven.
	\item De leerlingen kunnen door het doordacht experimenteren bepaalde vragen over combinaties van piramides onderzoeken.
	\item De leeringen kunnen een bewijs geven voor de som van de eerste $n$ kwadraten.
	\item De leerlingen kunnen op een analoge wijze het bewijs geven voor de som van de eerst $n$ driehoeksgetallen.
\end{itemize}

\subsubsection{Werkvormen}
\begin{itemize}
\item Onderwijsleergesprek.
\item In groepjes werken en experimenteren.
\end{itemize}

\subsubsection{Didactische wenken}
Na een korte inleiding van de leerkracht, wordt aan de leerlingen de opdracht gegeven om op zoek te gaan naar het aantal kanonskogels in een (regelmatige) piramide met vierkant en driehoekig grondvlak en waarbij de kanonskogels in $n$ lagen gestapeld zijn. Om dit uit te testen krijgen de leerlingen bijvoorbeeld knikkers of andere bolvormige voorwerpen. Ze kunnen eerst dit uittesten voor enkele kleine gevallen. Hieruit is het dan de bedoeling dat ze een formule vinden die afhangt van $n$, het aantal lagen van de piramide. De leerkracht loopt ondertussen rond en helpt waar mogelijk. Het is zeker de bedoeling dat de leerlingen vinden dat het de som is van de eerste $n$ kwadraten en eerste $n$ driehoeksgetallen.
Ook krijgen de leerlingen een aantal andere vragen mee die ze in kleine groepjes moeten gaan uitproberen. Uiteraard hoeven de leerlingen niet constant met de knikkers te werken. Eenmaal ze het systeem doorhebben, kunnen ze met de rijen op papier werken en algebra�sch de vragen oplossen.

Op het einde van de les is het de bedoeling dat de leerkracht met de leerling kijkt of ze de formule hebben gevonden en dat de leerlingen aan elkaar kunnen uitleggen hoe ze daaraan gekomen zijn. Daarna is het bedoeling dat de leerkracht met de leerlingen de expliciete formule geeft voor de eerste $n$ kwadraten. Op analoge wijze kunnen de leerlingen nu de formule van de eerste $n$ driehoeksgetallen vinden. Dit kan ook eventueel als opgave worden gegeven.

\subsection{Les 8: Vervolg kanonskogels en het bolstapelprobleem van Kepler}

\subsubsection{Lesdoelstellingen}

\begin{itemize}
\item De leerlingen kunnen door het doordacht experimenteren bepaalde vragen over combinaties van piramides onderzoeken.
\item De leerlingen kunnen vanuit een voorbeeld komen tot een algemene formule en dit ook bewijzen, al dan niet met behulp van de leerkracht.
	\item De leerlingen weten wat het bolstapelprobleem inhoudt.
	\item De leerlingen beseffen dat het vinden van een bewijs soms vele jaren kan duren.
	\item De leerlingen beseffen dat de controle van een bewijs soms geen evidentie is, bijvoorbeeld omdat een deel van het bewijs door een computerprogramma geleverd wordt.
	\item De leerlingen weten wat een \textquoteleft kissing number' is.
\end{itemize}

\subsubsection{Werkvormen}
\begin{itemize}
\item Onderwijsleergesprek.
\item Quiz.
\end{itemize}

\subsubsection{Didactische wenken}
We gaan in deze les verder met waar we de vorige les nog niet aan toe kwamen, namelijk het combineren van stapels van kanonskogels. De leerkracht overloopt samen met de leerlingen hun bevindingen over welke combinaties er mogelijk zijn, welke zeker niet, waarom niet enzovoort. Bij combinaties die mogelijk zijn wordt dan vanuit een voorbeeldcombinatie de algemene formule geconstrueerd. De gevonden formule wordt dan ook effectief bewezen met behulp van resultaten uit een vorige les.

Voor het tweede deel van de les is het de bedoeling dat we op een ludieke manier de evolutie van het bolstapelprobleem van Kepler schetsen. We doen dit aan de hand van een quiz. De leerlingen worden verdeeld in groepjes van ongeveer 4 personen en krijgen allemaal bordjes of briefjes met A, B, C en D op. Het is namelijk een quiz met meerkeuzevragen. Sommige antwoorden zouden de leerlingen tijdens de loop van deze lessenreeks al moeten gehoord hebben, andere vragen zijn eerder \textquoteleft gokvragen'. De leerkracht houdt ook de stand bij, voor elk goed antwoord krijgen de leerlingen ��n punt. Er zijn 8 vragen en bij elke vraag moeten de leerlingen op het teken van de leerkracht hun bordjes omhoog steken.
Deze quiz is echter maar een werkvorm om het verhaal van het bolstapelprobleem te kunnen aanbrengen.
Het is de bedoeling dat de leerlingen beseffen wat voor een evolutie achter bepaalde problemen zitten. En dat de uiteindelijke bewijzen maar pas geleverd werden met veel vallen en opstaan.
Wiskunde is iets dat leeft. Het zijn niet zozeer de namen van de wiskundigen die hier van belang zijn, maar eerder het proces hoe ze tot een oplossing zijn gekomen. Van het bedenken van nieuwe technieken en begrippen (bijvoorbeeld 
Voronoicellen) tot het reduceren van het probleem. Ook lopen de onderzoekers soms vast of zitten ze op een dood spoor (de verlaging van de bovengrens), maar ook in dat geval kan dit tot nieuwe ontdekkingen leiden.
De tekst die in de cursus staat is niet iets dat zomaar klakkeloos moet voorgelezen worden, maar is een leidraad voor de leerkracht om bij elke quizvraag iets te vertellen over het ontstaan, de evolutie en het bewijs.

\end{document}
