
\section{Oplossen van vergelijkingen}

Het vullen van vierkanten kan heel origineel gebruikt worden om een vergelijking op te lossen. 

\subsection{Oplossen van vierkantsvergelijkingen}

Een {\bf vierkantsvergelijking} is een vergelijking van de vorm $ax^2 + bx + c = 0$. Dus de hoogste graad van de onbekende is $2$. Jullie hebben reeds allemaal methoden hebben gezien om vergelijkingen van deze vorm op te lossen. De typische methoden zijn
\begin{itemize}
  \item Merkwaardige producten gebruiken.
  \item Ontbinden in factoren door delers te zoeken.
  \item De $abc$-formule gebruiken.
  \item Door kwadraatafsplitsing te gebruiken.
\end{itemize}

\ask{Welke van de bovenstaande methoden kennen jullie reeds?}

\answer[1cm]{}

\subsubsection{Een eenvoudig geval}

Een heel eenvoudige vierkantsvergelijking is
$$
x^2+2x+1=0\;.
$$

\ask{Wat is de eenvoudigste methode die je kan gebruiken? Los deze vierkantsvergelijking ook op.}

\answer[3cm]{We kunnen een merkwaardig product gebruiken om de veelterm te ontbinden in factoren: $x^2+2x+1=(x+1)^2$ Nadien kunnen we de oplossingen aflezen. De vergelijking heeft als oplossing twee keer $-1$, dus de oplossingsverzameling is $Opl=\{-1\}$.}

We zullen hier nu nog een methode aan toevoegen. Dit door een vierkant te construeren met kleinere vierkanten en rechthoeken. Hiervoor zullen we alle kwadraten zien als een vierkant, de andere getallen zullen we zien als rechthoeken. In ons bovenstaand voorbeeld hebben we als kwadraten alvast $x^2$ en $1$. We construeren twee vierkanten:

\begin{center}
\begin{tikzpicture}[scale=0.5, line cap=round,line join=round,>=triangle 45,x=1.0cm,y=1.0cm]
\clip(-1,-1) rectangle (9,6);
\filldraw[line width=1.6pt,fill=black,fill opacity=0.1] (0,0) -- (5,0) -- (5,5) -- (0,5) -- cycle;
\filldraw[line width=1.6pt,fill=black,fill opacity=0.1] (7,0) -- (8,0) -- (8,1) -- (7,1) -- cycle;
\draw (2,0) node[anchor=north west] {$x$};
\draw (-1,2.87) node[anchor=north west] {$x$};
\draw (7,0) node[anchor=north west] {$1$};
\draw (6,1) node[anchor=north west] {$1$};
\end{tikzpicture}

\end{center}

We hebben nog $2x$ over. Dit moeten we zien als een rechthoek. Een rechthoek van grootte $2$ bij $x$ helpt ons niets, want dan hebben we twee vierkanten en \'e\'en rechthoek die we niet kunnen samenvoegen om \'e\'en vierkant mee op te vullen. We zullen dus de overige $2x$ opsplitsen in $x + x$. Hiervan maken we dan twee rechthoeken met grootte $1$ bij $x$:

\begin{center}
\begin{tikzpicture}[scale=0.5, line cap=round,line join=round,>=triangle 45,x=1.0cm,y=1.0cm]
\clip(-1,-1) rectangle (5,6);
\filldraw[line width=1.6pt,fill=black,fill opacity=0.1] (0,0) -- (1,0) -- (1,5) -- (0,5) -- cycle;
\filldraw[line width=1.6pt,fill=black,fill opacity=0.1] (3,0) -- (4,0) -- (4,5) -- (3,5) -- cycle;
\draw (2,2.8) node[anchor=north west] {$x$};
\draw (-1,2.8) node[anchor=north west] {$x$};
\draw (0,0) node[anchor=north west] {$1$};
\draw (3,0) node[anchor=north west] {$1$};
\end{tikzpicture}

\end{center}

Nu zou het mogelijk moeten zijn om de figuren te verschuiven en te roteren zodanig dat je met de vier figuren \'e\'en groter vierkant kunt maken.

\task{Puzzel nu met de vier figuren tot je \'e\'en groter vierkant hebt. Desnoods kan je de figuren uitknippen uit een apart blad papier om hiermee te puzzelen. Teken hieronder dan het resultaat.}

\answer[5cm]{
\begin{center}
\begin{tikzpicture}[scale=0.5, line cap=round,line join=round,>=triangle 45,x=1.0cm,y=1.0cm]
\clip(-1,-1) rectangle (7,7);
\filldraw[line width=1.6pt,fill=black,fill opacity=0.1] (0,0) -- (5,0) -- (5,5) -- (0,5) -- cycle;
\filldraw[line width=1.6pt,fill=black,fill opacity=0.1] (5,0) -- (6,0) -- (6,5) -- (5,5) -- cycle;
\filldraw[line width=1.6pt,fill=black,fill opacity=0.1] (5,5) -- (6,5) -- (6,6) -- (5,6) -- cycle;
\filldraw[line width=1.6pt,fill=black,fill opacity=0.1] (0,5) -- (5,5) -- (5,6) -- (0,6) -- cycle;
\draw (2,0) node[anchor=north west] {$x$};
\draw (-1,2.8) node[anchor=north west] {$x$};
\draw (5,0) node[anchor=north west] {$1$};
\draw (-1,6) node[anchor=north west] {$1$};
\end{tikzpicture}

\end{center}
}

\ask{Wat is nu de oppervlakte van het bekomen vierkant? Kan je hiermee de vergelijking oplossen?}

\answer[3cm]{De oppervlakte is $(x+1)^2$. De oplossing is dan $Opl=\{-1\}$.}

Je kan analoog redeneren om de oplossingen te zoeken van elke vergelijking van de vorm $a^2x^2+2abx+b^2=0$. We zullen nu een methode zien waarbij we kwadratische vergelijkingen van de algemene vorm $ax^2+2bx+c=0$ kunnen oplossen.

\subsubsection{Een iets moeilijkere vergelijking}

We beginnen opnieuw met een vergelijking die mag opgelost worden met \'e\'en van de gekende methoden. Neem als vergelijking
$$
x^2+4x-5=0\;.
$$

Opnieuw zien we $x^2$ als een vierkant met zijde $x$. Ook zien we $4x$ terug als twee rechthoeken, beide met de zijden $x$ en $2$. We hebben dan al $x^2+4x=x^2+2x+2x$ waarbij de volgende figuur hoort.
\begin{center}
\begin{tikzpicture}[scale=0.5, line cap=round,line join=round,>=triangle 45,x=1.0cm,y=1.0cm]
\clip(-1,-1) rectangle (8,8);
\filldraw[line width=1.6pt,fill=black,fill opacity=0.1] (0,0) -- (5,0) -- (5,5) -- (0,5) -- cycle;
\filldraw[line width=1.6pt,fill=black,fill opacity=0.1] (5,0) -- (7,0) -- (7,5) -- (5,5) -- cycle;
\filldraw[line width=1.6pt,fill=black,fill opacity=0.1] (0,5) -- (5,5) -- (5,7) -- (0,7) -- cycle;
\draw (2.56,0) node[anchor=north] {$x$};
\draw (-0.5,2.87) node[anchor=north] {$x$};
\draw (6,0) node[anchor=north] {$2$};
\draw (-0.5,6.5) node[anchor=north] {$2$};
\end{tikzpicture}

\end{center}

Uit de vergelijking weten we dat $x^2+4x=5$. Dus de oppervlakte van deze figuur moet gelijk zijn aan $5$! Laten we deze figuur eens vervolledigen door rechtsboven een klein vierkantje van $2$ bij $2$ aan toe te voegen:

\begin{center}
\begin{tikzpicture}[scale=0.5, line cap=round,line join=round,>=triangle 45,x=1.0cm,y=1.0cm]
\clip(-1,-1) rectangle (8,8);
\filldraw[line width=1.6pt,fill=black,fill opacity=0.1] (0,0) -- (5,0) -- (5,5) -- (0,5) -- cycle;
\filldraw[line width=1.6pt,fill=black,fill opacity=0.1] (5,0) -- (7,0) -- (7,5) -- (5,5) -- cycle;
\filldraw[line width=1.6pt,fill=black,fill opacity=0.1] (0,5) -- (5,5) -- (5,7) -- (0,7) -- cycle;
\filldraw[line width=1.6pt,fill=black,fill opacity=0.1] (5,5) -- (7,5) -- (7,7) -- (5,7) -- cycle;
\draw (2.56,0) node[anchor=north] {$x$};
\draw (-0.5,2.87) node[anchor=north] {$x$};
\draw (6,0) node[anchor=north] {$2$};
\draw (-0.5,6.5) node[anchor=north] {$2$};
\end{tikzpicture}

\end{center}

De oppervlakte van deze nieuwe figuur is dus de oppervlakte van de vorige figuur plus de oppervlakte van het nieuwe vierkantje van $2$ bij $2$. De oppervlakte is dus $5 + 4 = 9$. De lengte van de zijde van dit vierkant is dus $\sqrt{9}=3$. Maar we wisten ook reeds dat de lengte van deze figuur $x+2$ was. We leiden dus gemakkelijk de waarde voor $x$ af.

\ask{Stel beide manieren om de lengte van de zijde te berekenen aan elkaar gelijk en los daar dan $x$ uit op.}

\answer[3cm]{
\begin{align*}
  3 &= x + 2\\
  \Leftrightarrow\quad 3-2 &= x\\
  \Leftrightarrow\quad x&=1
\end{align*}
}

We zien dat we in de figuur het vierkant $x^2$ te groot hebben getekend ten opzicht van de rechthoek van $x$ bij $2$ maar dat is niet erg. Het was maar een schets.

\ask{De vergelijking $x^2+4x-5=0$ heeft nog een tweede oplossing. Bepaal deze met een methode die je reeds kent (hint: aangezien $x=1$ een oplossing is, heb je reeds de deler $(x-1)$ van de veelterm $x^2+4x-5$ gevonden.) Waarom denk je dat het niet mogelijk is om door het vullen van een vierkant de andere oplossing te vinden?}

\answer[4cm]{We berekenen het quoti\"ent van $x^2+4x-5$ en $x-1$. We zien dat dit $x+5$ is. De andere oplossing voor $x$ is dus $-5$. Omdat negatieve getallen geen lengte voor kunnen stellen kan hiermee dan ook geen vierkant geconstrueerd worden.}

\subsubsection{Een algemene methode}

Bekijk nu eens de volgende vergelijking
$$
x^2-8x-20 = 0\;.
$$

\ask{Is het mogelijk om dezelfde techniek als voordien toe te passen? Waarom niet?}

\answer[3cm]{Neen. We kunnen geen rechthoeken construeren want de co\"effici\"ent bij de term van de eerste graad is negatief.}

Volg nu mee om deze vergelijking toch op te lossen.

\task{Teken een vierkant met zijde $x$.}

\answer[4cm]{
\begin{center}
\begin{tikzpicture}[scale=0.3, line cap=round,line join=round,>=triangle 45,x=1.0cm,y=1.0cm]
\clip(-1,-1) rectangle (11,11);
\filldraw[line width=1.6pt,fill=black,fill opacity=0.1] (0,0) -- (10,0) -- (10,10) -- (0,10) -- cycle;
\draw (5,0.25) node[anchor=north] {$x$};
\draw (-0.75,5) node[anchor=north] {$x$};
\end{tikzpicture}

\end{center}
}

Dit vierkant zal dus een oppervlakte van $x^2$ hebben.

\task{Teken nu in het vierkant van daarnet twee rechthoeken van 4 bij x. Deze moeten \uline{binnen in} het vierkant zelf getekend worden, bijvoorbeeld \'e\'en bovenaan, en \'e\'en rechts.}

\answer[4cm]{
\begin{center}
\begin{tikzpicture}[scale=0.3, line cap=round,line join=round,>=triangle 45,x=1.0cm,y=1.0cm]
\clip(-1,-1) rectangle (11,11);
\filldraw[line width=1.6pt,fill=black,fill opacity=0.1] (0,0) -- (10,0) -- (10,10) -- (0,10) -- cycle;
\filldraw[line width=1.6pt,fill=black,fill opacity=0.1] (6,0) -- (10,0) -- (10,10) -- (6,10) -- cycle;
\filldraw[line width=1.6pt,fill=black,fill opacity=0.1] (0,6) -- (10,6) -- (10,10) -- (0,10) -- cycle;
\draw (5,0.25) node[anchor=north] {$x$};
\draw (-0.75,5) node[anchor=north] {$x$};
\draw (8,1.75) node[anchor=north] {$4$};
\draw (1,8.5) node[anchor=north] {$4$};
\end{tikzpicture}

\end{center}
}

\ask{De twee rechthoeken overlappen elkaar. Wat is de oppervlakte van het overlappend stukje?}

\answer[1cm]{$4\cdot 4=16$.}

Linksonder blijft dan nog een stukje met oppervlakte $(x-4)^2$ over. Uit de figuur weten we dat dit gelijk is aan $x^2 -4x -4x$ plus het overlappend deel met oppervlakte $16$. En uit de vergelijking weet je dat $x^2 -4x -4x=20$ is.

\task{Stel beide manieren om de oppervlakte van het kleine vierkantje linksonder te berekenen aan elkaar gelijk en haal er zo een oplossing voor $x$ uit.}

\answer[3cm]{
\begin{align*}
  (x-4)^2 &= x^2 -4x -4x + 16\\
          &= 20 +16\\
          &= 36\\
\end{align*}
We kunnen nu de positieve wortel nemen want afstanden zijn positief, we krijgen dan:
$$ x-4 = 6\;.$$
We vinden dus een oplossing $x=10$. De andere oplossing kan bepaald worden door de tweedegraadsveelterm $x^2-8x-20$ te delen door $x-10$.
} 

Zo zie je maar. Een probleem uit de algebra kan gerust opgelost worden met een beetje meetkunde!













