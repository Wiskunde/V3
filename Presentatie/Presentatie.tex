\documentclass[dutch]{beamer}
\mode<presentation>

%\documentclass[handout]{beamer}
%\usepackage{pgfpages}
%\pgfpagesuselayout{4 on 1}[a4paper,border shrink=5mm,landscape]
%\setbeamertemplate{footline}[frame number]

\usepackage[dutch]{babel}
\usepackage{graphicx}
\usepackage{amssymb, subfigure}
\usepackage{amsthm}
\usepackage[utf8]{inputenc}
\usepackage[normalem]{ulem}
\usepackage{color}
\usepackage{cancel}
%\usepackage{translator}

\swapnumbers\newtheorem{stelling}{Stelling}[section]
\newtheorem{hulpstelling}[stelling]{Hulpstelling}
\newtheorem{gevolg}[stelling]{Gevolg}
\newtheorem{gevolgen}[stelling]{Gevolgen}
\newtheorem{voorbeeld}[stelling]{Voorbeeld}
\newtheorem{voorbeelden}[stelling]{Voorbeelden}
\newtheorem{belangrijkvoorbeeld}[stelling]{Belangrijk voorbeeld}
\newtheorem{belangrijkgevolg}[stelling]{Belangrijk gevolg}
\newtheorem{belangrijkegevolgen}[stelling]{Belangrijke gevolgen}
\newtheorem{opmerking}[stelling]{Opmerking}
\newtheorem{opmerkingen}[stelling]{Opmerkingen}
\newtheorem{definitie}[stelling]{Definitie}
\newtheorem{toepassing}[stelling]{Toepassing}
\newtheorem{oefeningen}[stelling]{Oefeningen}
\newtheorem{definities}[stelling]{Definities}
\newtheorem{oefening}[stelling]{Oefening}
\newtheorem{notatie}[stelling]{Notatie}
\newtheorem{onderzoeksvraag}{Onderzoeksvraag}

\graphicspath{{afbeeldingen/}}

%\usetheme{Copenhagen}
%\usecolortheme{dolphins}
%rgb ---> 0.3,0.5,0.7 mooi blauw
\usetheme{antibes}
%\definecolor{mycolor}{RGB}{48,96,102}
%\definecolor{mycolor2}{RGB}{63,125,133}
%\setbeamercolor*{palette primary}{use=structure,fg=white,bg=mycolor}
\setbeamertemplate{navigation symbols}{}



\title[Vullen van Vlakken en Volumes]{Vullen van Vlakken en Volumes}
\author[Lien Lambert \& Pieter Pareit \& Jordy Vanpoucke ]{Lien Lambert \\Pieter Pareit \\Jordy Vanpoucke}
\date{\tiny 15 mei 2012}

\begin{document}

\frame{
%\begin{figure}[t]
%	\flushleft
%		\includegraphics[scale=0.1]{logo.jpg}
%\end{figure}
\titlepage}

\frame{
\tableofcontents 
}

\section{Inleiding}
\subsection{Motivatie}
\begin{frame}
\frametitle{Waarom?}
\pause
\begin{figure}[h]
\centering
\only<2>{\includegraphics[height=4.8cm]{oranges}}
\pause
\only<3>{\includegraphics[height=4.8cm]{cannonballs}}
\pause
\only<4>{\includegraphics[height=4.8cm]{dozen}}
\pause
\only<5>{\includegraphics[height=4.8cm]{dienblad}}
\pause
\only<6>{\includegraphics[height=4.8cm]{jetons_hex}}
\end{figure}
\end{frame}


\subsection{Doelgroep en voorkennis}

\begin{frame}
\frametitle{Doelgroep}
\pause
\begin{itemize}
\item $2^{de}$ graad ASO
\pause
\item Ook $3^{de}$ graad ASO\pause, maar \ldots
\end{itemize}
\end{frame}


\begin{frame}
\frametitle{Voorkennis}
\pause
\begin{itemize}
\item verschillende vakgebieden
\pause
\begin{itemize}
\item getallenleer en algebra
\item re\"{e}le functies
\item meetkunde
\end{itemize}
\pause
\item `voorkennis'
\end{itemize}
\end{frame}

\section{Overzicht van de lessen.}

\begin{frame}
\begin{center}
\textbf{\Large{Vullen van Vlakken en Volumes}}
\end{center}
\end{frame}
\subsection{Les 1: Inleiding}
\begin{frame}
\begin{itemize}
\item De leerlingen kennen de definities van een vlakvulling, een complete vlakvulling en een optimale vlakvullling.
\item De leerlingen kunnen in groep komen tot bevindingen over het compleet vullen van een oneindig groot vlak.
\item De leerlingen kunnen bewijzen dat een betegeling van een oneindig groot vlak met regelmatige veelhoeken enkel lukt met regelmatige driehoeken, regelmatige vierhoeken en regelmatige zeshoeken.
\item De leerlingen kennen stapelproblemen.
\item De leerlingen kunnen zelfstandig op zoek gaan naar meer informatie over het lesonderwerp.
\item De leerlingen kunnen in groep samenwerken om enkele raadsels op te lossen.
\end{itemize}
\end{frame}

\subsection{Les 2: Vullen van vierkanten met kleinere vierkanten en zo tot Perfecte Vierkanten komen}

\begin{frame}
\begin{itemize}
\item De leerlingen kunnen een een vierkant met kleinere vierkanten vullen van de zelfde grootte, om zo een algemene formule die de maximale effici\"{e}ntie bepaald, te vinden.
\item De leerlingen weten wat een perfect vierkant van orde $n$ is.
\item De leerlingen kunnen een ondergrens zoeken voor de orde van het perfect vierkant door redeneren.
\item De leerlingen zien in dat een perfect vierkant dat gedraaid of gespiegeld wordt geen ander perfect vierkant oplevert. Hiervoor leren de leerlingen het begrip isomorfisme.
\item De leerlingen beseffen dat het vinden van \'{e}\'{e}n perfect vierkant leidt tot het vinden van een oneindig aantal perfecte vierkanten. 
\item De leerlingen kunnen een bewijs uit het ongerijmde opstellen om een eigenschap te bewijzen over perfecte vierkanten.
\item De leerlingen kunnen hun resultaten en definities in verband met perfecte vierkanten uitbreiden naar \'{e}\'{e}n en drie dimensies.
\end{itemize}
\end{frame}



\subsection{Les 3: Puzzelen met vierkanten en rechthoeken om vergelijkingen op te lossen}

\begin{frame}
\begin{itemize}
  \item De leerlingen kunnen voor hen gekende methoden, uit de algebra, gebruiken om problemen op te lossen, i.h.b. de methode voor het oplossen van een vierkantsvergelijking.
  \item De leerlingen kunnen dan dezelfde problemen op een meetkundige manier oplossen.
  \item De leerlingen beseffen dat er een link is tussen algebra en meetkunde.
  \item De leerlingen hebben een idee hoe het oplossen van derdegraadsvergelijkingen historisch is gegroeid.
\end{itemize}
\end{frame}

\subsection{Les 4: Kennismaking met tweedimensionale stapelproblemen en hiermee experimenteren}

\begin{frame}
\begin{itemize}
\item De leerlingen beseffen dat je een tafel niet volledig kan bedekken met cirkels en dat we daarom op zoek kunnen gaan naar de optimale configuratie, namelijk het hexagonaal rooster.
\item De leerlingen weten wat een voronoicel is en kunnen hiermee de effici\"{e}ntie van een rooster berekenen.
\item De leerlingen kunnen de oppervlakte van een driehoek, rechthoek, zeshoek en cirkel berekenen.
\item De leerlingen kunnen a.d.h.v. een geogebra-applet of door te experimenteren met jetons of munten de effici\"{e}ntie van dienbladen bepalen.
\item De leerlingen kunnen de effici\"{e}ntie voor eenvoudige voorbeelden uitrekenen.
\item De leerlingen ontwikkelen meetkundig inzicht.
\end{itemize}
\end{frame}

\subsection{Les 5: Het dienbladenprobleem: experimenteren en effectief berekenen}

\begin{frame}
\begin{itemize}
\item De leerlingen kunnen a.d.h.v. een Geogebra-applet of door te experimenteren met jetons(of munten) de effici\"{e}ntie van dienbladen bepalen.
\item De leerlingen kunnen functioneel gebruik maken van ICT.
\item De leerlingen kunnen de effici\"{e}ntie voor eenvoudige voorbeelden uitrekenen.
\item De leerlingen kunnen vanuit een figuur meetkundig inzicht verwerven en hierdoor problemen vereenvoudigen.
\item De leerlingen kunnen de meetkundige interpretatie van de goniometrische formules functioneel gebruiken, i.h.b. voor projecties.
\end{itemize}
\end{frame}



\subsection{Les 6: Het dienbladenprobleem: berekenen, bewijzen en besluiten}

\begin{frame}
\begin{itemize}
\item De leerlingen beseffen dat men uit experimenteren belangrijke informatie kan halen en beseffen ook het belang van bewijzen.
\item De leerlingen kunnen aan de hand van een stappenplan het bewijs zelf verwerven.
\item De leerlingen kunnen de effici\"{e}ntie van ingewikkelde configuraties berekenen.
\item De leerlingen kunnen vorige berekeningen gebruiken om ingewikkelde configuraties te vereenvoudigen.
\item De leerlingen kunnen uit experimentele resultaten conclusies trekken over de meest effici\"{e}nte dienbladen, en kunnen ook (mogelijke) verklaringen geven.
\end{itemize}
\end{frame}

\subsection{Les 7: Kanonskogels stapelen}

\begin{frame}
\begin{itemize}
\item De leerlingen weten wat het bolstapelprobleem inhoudt.
\item De leerlingen kunnen het aantal kanonskogels in een (regelmatige) piramide met vierkant en driehoekig grondvlak met $n$ lagen experimenteel berekenen.
\item De leerlingen kunnen vanuit een gegeven rij het recursief en/of expliciet voorschrift geven.
\item De leerlingen kunnen door het doordacht experimenteren bepaalde vragen over combinaties van piramides onderzoeken.
\item De leeringen kunnen een bewijs geven voor de som van de eerste $n$ kwadraten.
\item De leerlingen kunnen op een analoge wijze het bewijs geven voor de som van de eerst $n$ driehoeksgetallen.
\end{itemize}
\end{frame}

\subsection{Les 8: Vervolg kanonskogels en het bolstapelprobleem van Kepler}

\begin{frame}
\begin{itemize}
\item De leerlingen kunnen door het doordacht experimenteren bepaalde vragen over combinaties van piramides onderzoeken.
\item De leerlingen kunnen vanuit een voorbeeld komen tot een algemene formule en dit ook bewijzen, al dan niet met behulp van de leerkracht.
\item De leerlingen weten wat het bolstapelprobleem inhoudt.
\item De leerlingen beseffen dat het vinden van een bewijs soms vele jaren kan duren.
\item De leerlingen beseffen dat de controle van een bewijs soms geen evidentie is, bijvoorbeeld omdat een deel van het bewijs door een computerprogramma geleverd wordt.
\item De leerlingen weten wat een `kissing number' is.
\end{itemize}
\end{frame}


\end{document}


