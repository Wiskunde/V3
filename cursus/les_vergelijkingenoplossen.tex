
\section{Oplossen van vergelijkingen}

Het vullen van vierkanten kan heel origineel gebruikt worden om een vergelijking op te lossen. 

\subsection{Oplossen van vierkantsvergelijkingen}

Een {\bf vierkantsvergelijking} is een vergelijking van de vorm $ax^2 + bx + c = 0$. Dus de hoogste graad van de onbekende is $2$. Jullie hebben reeds allemaal methoden hebben gezien om vergelijkingen van deze vorm op te lossen. De typische methoden zijn
\begin{itemize}
  \item Merkwaardige producten gebruiken.
  \item Ontbinden in factoren door delers te zoeken.
  \item De $abc$-formule gebruiken.
  \item Door kwadraatafsplitsing te gebruiken.
\end{itemize}

\ask{Welke van de bovenstaande methoden kennen jullie reeds?}

\answer[1cm]{}

Een heel eenvoudige vierkantsvergelijking is
$$
x^2+2x+1=0
$$

\ask{Wat is de eenvoudigste methode die je kan gebruiken? Los deze vierkantsvergelijking ook op.}

\answer[3cm]{We kunnen een merkwaardig product gebruiken om de veelterm te ontbinden in factoren: $x^2+2x+1=(x+1)^2$ Nadien kunnen we de oplossingen aflezen. De vergelijking heeft als oplossing twee keer $-1$, dus de oplossingsverzameling is $Opl=\{-1\}$.}

We zullen hier nu nog een methode aan toevoegen. Dit door een vierkant te construeren met kleinere vierkanten en rechthoeken. Hiervoor zullen we alle kwadraten zien als een vierkant, de andere getallen zullen we zien als rechthoeken. In ons bovenstaand voorbeeld hebben we als kwadraten alvast $x^2$ en $1$. We construeren twee vierkanten:

\begin{center}
\begin{tikzpicture}[scale=0.5, line cap=round,line join=round,>=triangle 45,x=1.0cm,y=1.0cm]
\clip(-1,-1) rectangle (9,6);
\filldraw[line width=1.6pt,fill=black,fill opacity=0.1] (0,0) -- (5,0) -- (5,5) -- (0,5) -- cycle;
\filldraw[line width=1.6pt,fill=black,fill opacity=0.1] (7,0) -- (8,0) -- (8,1) -- (7,1) -- cycle;
\draw (2,0) node[anchor=north west] {$x$};
\draw (-1,2.87) node[anchor=north west] {$x$};
\draw (7,0) node[anchor=north west] {$1$};
\draw (6,1) node[anchor=north west] {$1$};
\end{tikzpicture}

\end{center}

We hebben nog $2x$ over. Dit moeten we zien als een rechthoek. Een rechthoek van grootte $2$ bij $x$ helpt ons niets, want dan hebben we twee vierkanten en \'e\'en rechthoek die we niet kunnen samenvoegen om \'e\'en vierkant mee op te vullen. We zullen dus de overige $2x$ opsplitsen in $x + x$. Hiervan maken we dan twee rechthoeken met grootte $1$ bij $x$:

\begin{center}
\begin{tikzpicture}[scale=0.5, line cap=round,line join=round,>=triangle 45,x=1.0cm,y=1.0cm]
\clip(-1,-1) rectangle (5,6);
\filldraw[line width=1.6pt,fill=black,fill opacity=0.1] (0,0) -- (1,0) -- (1,5) -- (0,5) -- cycle;
\filldraw[line width=1.6pt,fill=black,fill opacity=0.1] (3,0) -- (4,0) -- (4,5) -- (3,5) -- cycle;
\draw (2,2.8) node[anchor=north west] {$x$};
\draw (-1,2.8) node[anchor=north west] {$x$};
\draw (0,0) node[anchor=north west] {$1$};
\draw (3,0) node[anchor=north west] {$1$};
\end{tikzpicture}

\end{center}

Nu zou het mogelijk moeten zijn om de figuren te verschuiven en te roteren zodanig dat je met de vier figuren \'e\'en groter vierkant kunt maken.

\task{Puzzel nu met de vier figuren tot je \'e\'en groter vierkant hebt. Desnoods kan je de figuren uitknippen uit een apart blad papier om hiermee te puzzelen. Teken hieronder dan het resultaat.}

\answer[5cm]{
\begin{center}
\begin{tikzpicture}[scale=0.5, line cap=round,line join=round,>=triangle 45,x=1.0cm,y=1.0cm]
\clip(-1,-1) rectangle (7,7);
\filldraw[line width=1.6pt,fill=black,fill opacity=0.1] (0,0) -- (5,0) -- (5,5) -- (0,5) -- cycle;
\filldraw[line width=1.6pt,fill=black,fill opacity=0.1] (5,0) -- (6,0) -- (6,5) -- (5,5) -- cycle;
\filldraw[line width=1.6pt,fill=black,fill opacity=0.1] (5,5) -- (6,5) -- (6,6) -- (5,6) -- cycle;
\filldraw[line width=1.6pt,fill=black,fill opacity=0.1] (0,5) -- (5,5) -- (5,6) -- (0,6) -- cycle;
\draw (2,0) node[anchor=north west] {$x$};
\draw (-1,2.8) node[anchor=north west] {$x$};
\draw (5,0) node[anchor=north west] {$1$};
\draw (-1,6) node[anchor=north west] {$1$};
\end{tikzpicture}

\end{center}
}

\ask{Wat is nu de oppervlakte van het bekomen vierkant? Kan je hiermee de vergelijking oplossen?}

\answer[3cm]{De oppervlakte is $(x+1)^2$. De oplossing is dan $Opl=\{-1\}$.}
